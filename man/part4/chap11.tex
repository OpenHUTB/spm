\chapter{SPM用于脑电图和脑磁图概述}


\section{欢迎使用SPM进行脑磁图和脑电图分析}

SPM在脑磁图和脑电图数据分析中的功能由三个主要部分组成。


基于体素图像的统计分析。
对于统计分析,我们使用与 SPM fMRI 用户相同的常规方法。
这些方法是基于广义线性模型和随机场理论的稳健且验证过的函数。
这些统计方法同样适用于多主体(或单主体)脑磁图和脑电图研究。


源重建。
我们团队在建立贝叶斯方法用于脑磁图和脑电图数据的源重建方面投入了大量资源。
良好的源重建技术对于脑磁图和脑电图领域至关重要,否则很难将传感器数据与神经解剖学或其他模态(如 fMRI)的发现联系起来。
贝叶斯源重建提供了一种系统化的方式来整合关于数据生成方式的先验知识,并使模型比较方法更加系统化。
通过使用先验信息和贝叶斯模型比较,脑磁图和脑电图源重建成为一种非常强大的神经成像工具,具有对神经元动态的独特宏观视角。


动态因果建模是一种时空网络模型,用于估计源网络中的有效连接性。
对于脑磁图和脑电图来说,动态因果建模是一种强大的技术,因为这些数据在时间上具有高分辨率,这使得神经生物学启发的网络模型的可识别性变得可行。
这意味着动态因果建模能够对源的时间先后关系进行推断,并可以量化源之间前馈、反馈和横向连接在毫秒级别神经时间尺度上的变化。


为了使用户能够为 SPM 分析准备他们的数据,我们还实现了一系列工具,覆盖从脑磁图或脑电图设备获取原始数据开始的完整分析流程。

我们的总体目标是提供一个学术性脑磁图和脑电图分析软件包,使每个人都能应用最新的脑磁图和脑电图数据分析方法。
尽管 SPM 的开发专注于由我们团队首创的一些特定方法,但我们致力于使用户能够轻松地在 SPM 和其他软件包之间进行数据处理。
我们与优秀的 FieldTrip 软件包(主要开发者:Robert Oostenveld,位于荷兰奈梅亨的 F.C. Donders 中心)在许多分析问题上进行正式合作。
例如,SPM 和 FieldTrip 共享将数据转换为 Matlab 格式的常规方法、用于脑磁图和脑电图源重建的前向建模,并且 SPM 发行版中包含了 FieldTrip 的一个版本,因此用户可以在自定义脚本中结合使用 FieldTrip 和 SPM 的功能。
SPM 和 FieldTrip 互补性很好,因为 SPM 专注于特定的分析工具,而 FieldTrip 是一个灵活的不同方法的综合库,可以灵活地组合来执行各种分析。
然而,这种灵活性使 FieldTrip 对于非专业用户来说不易上手。
FieldTrip 没有图形用户界面,其功能通过编写自定义的 Matlab 脚本来使用。
通过结合 SPM 和 FieldTrip 的优势,FieldTrip 的灵活性可以通过 SPM 的图形用户界面工具和批处理系统来补充。
在此框架内,高级用户可以轻松快速地开发带有图形用户界面的专用分析工具,这些工具也可以供不熟悉 Matlab 的用户使用。
在 SPM 附带的 MEEG 工具箱中,有一些这样的工具可供使用。
我们也乐意将其他用户贡献的新工具纳入这个工具箱,只要这些工具具有普遍兴趣和适用性。


\section{从SPM8到SPM12的变化}

SPM8对SPM5中最初实现的脑磁图和脑电图分析进行了重大变更。
主要的变化是使用了一种不同的数据格式,该格式使用对象来确保数据结构的内部一致性和完整性,并为使用脑磁图和脑电图数据的函数提供了一致的接口。
对象的使用显著提高了 SPM 代码的稳定性和鲁棒性。
从 SPM8 到 SPM12 的数据格式和对象细节的变化相对较小。
这些变化的目的是合理化内部数据结构和对象方法,以消除一些‘历史性的’设计错误和不一致之处。
例如,SPM8 中的 meegchannels、eogchannels 和 ecgchannels 方法已被 indchantype 方法所取代,后者接受所需的通道类型作为参数并返回通道索引。
indchantype 是几个具有类似功能的方法之一,其他包括 indsample、indchannel、indtrial(取代了 pickconditions)和 indfrequency。

数据预处理功能的另一个重大变化是移除了交互式图形用户界面元素,转而使用 SPM 的批处理系统。
这使得构建用于执行完整复杂数据分析的处理流程变得更加容易,而无需编程。
使用批处理具有许多优点,但也可能使某些操作变得复杂,因为批处理必须提前配置,不能依赖于输入文件中可用的信息。
例如,批处理工具无法知道特定数据集的通道名称,因此不能生成对话框供用户选择通道。
为简化需要这种信息的处理步骤,在‘批处理输入’菜单下的‘准备’工具中增加了额外的功能。
现在可以使用交互式图形用户界面为特定数据集做出必要的选择,然后将结果保存为 mat 文件,并将该文件用作批处理的输入。

接下来的章节将介绍 SPM 中所有与脑电图和脑磁图相关的功能。
大多数用户可能会发现教程(第\ref{ch:40}章)对于快速入门很有用。
关于数据转换、预处理功能和显示的详细描述在第\ref{ch:12}章中提供。
在第\ref{ch:13}章,我们解释了如何使用 SPM 的统计机制来分析脑磁图和脑电图数据。
第\ref{ch:14}章描述了 3D 源重建的常规方法,包括偶极子建模。
最后,在第\ref{ch:16}章中,我们介绍了动态因果建模的图形用户界面,包括对诱发反应、引发反应和局部场电位的建模。
