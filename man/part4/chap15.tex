\chapter{等效电流偶极子的定位}

本章描述了基于“变分贝叶斯等效电流偶极子”(VB-ECDs)的源重建方法。
有关实现的更多详细信息,请参阅相关例程中的帮助和注释,以及原始论文。


\section{简介}

3D成像(或分布式)重建方法同时考虑所有可能的源位置,允许大范围的活动簇与活动相对集中的“等效电流偶极子”(ECD)方法形成对比。
ECD方法依赖于两个假设:

·仅有少数(如少于5个)源同时处于活动状态,

·这些源非常集中。

这导致了ECD模型,即通过位于脑体积内的少数离散电流源(即偶极子)来解释观测到的头皮电位。

与3D成像重建相比,ECD模型中考虑的ECD数量(即“活跃位置”数量)需要预先定义。
这是一个关键步骤,因为所考虑的源数量定义了ECD模型。
此选择应基于对所观察到的脑活动的经验知识或任何其他信息来源(例如,通过观察头皮电位分布)。
通常,每个偶极子由6个参数描述:3个位置参数、2个方向参数和1个幅度参数。
一旦固定了ECD的数量,非线性优化算法将用于调整偶极子的参数(偶极子数量的6倍)以匹配观测到的电位。

传统的ECD方法使用“最小平方误差”标准进行简单的最佳拟合优化。
这导致了相对简单的算法,但也存在一些缺点:

·在框架中难以包含对偶极子的约束;

·噪声不能被适当地考虑,因为它的方差应与偶极子参数一起估计;

·难以定义估计参数的置信区间,这可能导致对结果的过度自信解释;

·只能通过拟合优度来比较不同数量偶极子的模型,而这可能是误导性的。

由于将偶极子添加到模型中必然会改善整体拟合优度,因此有人可能会错误地尝试使用尽可能多的ECD来完美拟合观测信号。
然而,通过使用贝叶斯技术,可以规避上述所有传统方法的局限性。

简要地说,构建了一个概率生成模型,为数据提供了似然模型。
该模型由一组参数先验组成,形成了一个贝叶斯模型,允许包含用户指定的先验约束。

然后,采用“变分贝叶斯”(VB)方案,通过迭代过程估计参数的后验分布。
因此,通过估计的参数后验方差,可以直接获得估计参数的置信区间。
在贝叶斯背景下,可以使用模型证据或边际似然来比较不同的模型。
这种模型比较优于传统的拟合优度度量,因为它考虑了模型的复杂性(例如偶极子的数量),并且隐含地考虑了模型参数的不确定性。
因此,VB-ECD可以为以下问题提供客观且准确的答案:该数据集由2个还是3个ECD更好地建模?


\section{在 SPM12 中的过程}

本节旨在描述如何在 SPM12 中使用 VB-ECD 方法。


\subsection{头部和前向模型}

用于计算偶极源在头皮电极上的投影的引擎来自 Fieldtrip,3D 成像或 DCM 使用的也是相同的引擎。
因此,头部模型的准备方式应该相同,如第14章所述。
因此,对于同一数据集,VB-ECD 和成像重建之间的差异仅归因于重建方法的不同。


\subsection{VB-ECD 重建}

在加载并准备好头部模型后,按下“反演”按钮以开始。
首先会看到“成像”、“VB-ECD”和“DCM”选项。
“成像”重建对应于第\ref{ch:14}章所述的成像解决方案,“DCM”在第\ref{ch:16}章中描述。
然后需要填写有关 ECD 模型的信息,并按以下顺序点击按钮:

1.指定重建的时间段或时间窗口,时间范围在整个时段内。
请注意,数据将在所选时间窗口内进行平均!
因此,VB-ECD 总是针对单一时间段计算。

2.输入需要重建的试验类型。
每种试验类型将单独重建。

3.向模型中添加单个(即独立的)偶极子或一对对称偶极子。
每个“元素”(单个或成对)单独添加到模型中。

4.使用“信息性”或“非信息性”位置先验。
“非信息性”表示在整个脑体积上平坦的先验。
“信息性”允许你输入源的先验位置。

5.使用“信息性”或“非信息性”矩先验。
“非信息性”表示对所有可能的方向和幅度采用平坦的先验。
“信息性”允许你输入源的先验矩。

6.返回步骤3,向模型中添加更多的偶极子,或者停止添加偶极子。

7.指定迭代次数。
这些是使用不同初始条件重复拟合过程的次数。
由于目标函数中存在多个局部最大值,特别是在选择非信息性位置先验时,多个迭代是获得良好结果所必需的。

此程序随后使用 VB 优化方案来估计模型参数。
在中间结果上有图形显示。
当选择最佳解决方案时,模型证据将显示在 SPM 图形窗口的顶部。
此数值可用于比较具有不同先验条件的解决方案。

最终结果将保存在数据结构 D 中的 .inv{D.val}.inverse 字段中,并显示在图形窗口中。


\subsection{结果显示}

最新的 VB-ECD 结果可以通过函数 D = spm\_eeg\_inv\_vbecd\_disp 再次显示。
如果需要显示特定的重建结果,可以使用:spm\_eeg\_inv\_vbecd\_disp('Init', D, ind)。
在 GUI 中,可以使用 “dip” 按钮(位于 “Invert” 按钮下方)显示偶极子位置。

在上部,三个主图显示了大脑的三个正交视图,并在上面叠加了偶极子的位置和方向。
位置置信区间通过三个视图中的偶极子位置周围的虚线椭圆描述。
由于显示自动集中在显示的偶极子上,因此无法点击图像。
不过,可以通过右键上下文菜单放大图像。

左下角的表格显示了当前偶极子的位置、方向(笛卡尔坐标或极坐标)和振幅,以各种格式显示。

右下角的表格允许选择试验类型和偶极子。
也可以显示多个试验类型和多个偶极子。
显示将集中在偶极子平均位置上。