\chapter{三维源重建:成像方法}
\label{ch:14}

本章描述了一种用于3D源重建的成像方法。


\section{引言}

本章重点介绍在 SPM 中实现 EEG/MEG 源重建的成像(或分布式)方法。
这种方法将传感器数据空间投影到(3D)大脑空间,并将大脑活动视为由分布在皮层上的大量偶极源组成,这些源具有固定的位置和方向。
这样就使得观测模型变得线性,未知变量是源的振幅或功率。

给定分段和预处理的数据(见第\ref{ch:12}章),可以估计每个偶极源在单个时间样本或更广的刺激时间窗口内的诱发和/或诱导活动。

获得的重建活动在3D体素空间中,可以进一步使用SPM中的大规模单变量分析进行分析。

与PET/fMRI数据重建不同,EEG/MEG源重建是一个非平凡的操作。
它通常被比作从阴影估计身体形状,从头皮数据推断大脑活动在数学上是不适定的,需要先验信息,如解剖、功能或数学约束,以隔离出唯一的和最可能的解决方案。

分布式线性模型已经存在了几十年,SPM中提出的成像解决方案流程是经典的,并且与该领域常见的方法非常相似。
然而,至少有两个方面是相当独特的,值得在这里强调:

·基于经验贝叶斯形式主义的反演旨在具有通用性,因为它可以结合并估计各种性质的多个约束的相关性;通过贝叶斯模型比较使数据驱动的相关性估计成为可能。

·在数据生成模型中结合了受试者的特定解剖结构,以一种避免个体皮层表面提取的方式。
个体皮层网格是从MNI空间中的标准网格自动获得的,这提供了一种在立体坐标中报告结果的简单高效的方法。

EEG/MEG 成像流程分为四个连续的步骤,外加一个总结结果的步骤。
这些步骤定义了任何反演过程。
本章将逐步介绍进行完整反演分析时需要完成的各个步骤:

1.源空间建模

2.数据共注册

3.正向计算

4.反向重建

5.总结反向重建结果为图像

前三个步骤是整个生成模型的一部分,而反向重建步骤包括贝叶斯反演,并且是唯一涉及实际 EEG/MEG 数据的步骤。


\section{入门}

以下内容可以通过选择“EEG”应用程序中的3D源重建按钮从SPM用户界面访问。
当您按下此按钮时,将出现一个新的窗口,带有引导您完成数据成像重建必要步骤的GUI。
在每个步骤中,与该步骤无关的按钮将被禁用。
当您打开窗口时,唯一可按的两个按钮是“Load”(加载预处理的SPM MEEG数据集)和“Group inversion”(组反演),后者将在下面描述。
您可以加载以下几种类型的数据集:包含不同条件的单次试验的分段数据集、每个条件有一个事件相关电位(ERP)的平均数据集,或总体平均数据集。
加载数据集的重要前提是数据集中必须包含传感器和基准点。
这将在您加载文件时进行检查,如果有问题则加载会失败。
确保数据集中每种模态(EEG或MEG)都有传感器描述。
例如,如果您有一个包含一些不想用于源重建的EEG通道的MEG数据集,请在加载数据集之前将它们的类型更改为“LFP”或“Other”(区别在于LFP通道仍会被滤波并可用于伪影检测,而Other通道则不会)。
通过SPM从其原始格式转换的MEG数据集将始终包含传感器和基准点描述。
对于某些支持的EEG通道设置(如扩展的10-20系统或BioSemi),SPM将提供默认的通道位置和基准点,可用于您的重建。
传感器和基准点描述可以使用Prepare界面进行修改,在此界面中,您还可以通过执行共注册来验证这些描述是否合理(更多关于共注册的详细信息见第\ref{ch:12}章及下文)。

成功加载数据集后,系统会要求您为当前分析单元命名。
在SPM中,可以使用不同参数对同一数据集进行多次重建。
如果按下“Save”按钮,这些重建结果将与数据集一起保存。
可以使用3D GUI和SPM EEG Review工具再次加载和查看这些结果。
从命令行可以通过meeg对象的D.inv字段访问源重建结果。
如果存在,这个字段是一个结构数组,不需要方法即可访问和修改。
每个单元包含不同重建的结果。
在GUI中,您可以使用第二行的按钮在这些单元之间导航,还可以创建、删除和清除单元。
输入的标签将附加到单元上以便您识别。


\section{源空间建模}

在输入标签后,您会看到“Template”和“MRI”按钮被启用。
“MRI”按钮将根据受试者的结构扫描创建描述不同头部腔室边界的个体头部网格。
SPM将要求提供受试者的结构图像。
由于图像需要分割,准备模型可能需要一些时间。
个体网格是通过将标准化个体结构图像到MNI模板所需的变形场的逆应用于从该模板衍生的标准网格来生成的。
这种方法比直接从结构图像衍生网格更稳健,即使个体结构图像质量较低也能工作。

目前,我们建议使用EEG的“Template”按钮和基于个体结构扫描的MEG头部模型。
如果没有个体结构扫描,将模板头部模型与个体头部形状结合也会得到相当精确的头部模型。
模板按钮使用基于MNI大脑的SPM模板头部模型。
相应的结构图像可以在SPM目录下的canonical$\backslash$single\_subj\_T1.nii中找到。
使用模板时,根据数据是EEG还是MEG,不同的事情会发生。
对于EEG,电极位置将被变换以匹配模板头部。
因此,即使受试者的头部与模板有很大不同,也应该能够获得良好的结果。
对于MEG,模板头部将被变换以匹配MEG数据中包含的基准点和头部形状。
在这种情况下,头部形状测量有助于SPM提供更多数据以正确缩放头部。
从用户的角度来看,这两种选项看起来非常相似。

无论使用MRI按钮还是Template按钮,描述可能的EEG和MEG信号源位置的皮层网格都是从模板网格获得的。
对于EEG,网格按原样使用,而对于MEG,网格会随头部模型变换。
有三种皮层网格大小可供选择:“粗略”、“正常”和“精细”(分别为5124、8196和20484个顶点)。
建议使用“正常”网格。
如果您的计算机难以处理“正常”选项,请选择“粗略”。
“精细”选项只适用于64位系统,并且可能有些过于复杂。


\section{共注册}

为了使 SPM 能够对源重建结果进行有意义的解释,必须将传感器位置最初表示的坐标系链接到结构 MRI 图像的坐标系(MNI 坐标)。
通常,为了在两个坐标系之间建立链接,您需要一组至少三个在两个系统中已知坐标的点。
这种点集就像是一种“罗塞塔石碑”,可以用来将任意点的位置从一个系统转换到另一个系统。
这些点称为“基准点”,为 SPM 提供创建数据“罗塞塔石碑”的所有必要信息的过程称为“共注册”。

将EEG/MEG数据共注册到结构MRI空间有两种可能的方法:

1.基于地标的共注册(仅使用基准点):
刚性变换矩阵(旋转和平移)被计算出来,以便在EEG/MEG空间中的每个基准点与sMRI空间中的对应基准点匹配。
然后将相同的变换应用于传感器位置。

2.表面匹配(在MEG/EEG空间中的某个头部形状与从sMRI派生的头皮镶嵌之间进行匹配):
对于EEG,可以使用传感器位置代替头部形状。
对于MEG,首先将头部形状共注册到sMRI空间;然后将逆变换应用于头部模型和网格。
表面匹配是使用迭代最近点算法(ICP)进行的。
ICP算法是一种迭代对齐算法,分三个阶段进行:

·建立两个结构中要对齐的特征对之间的对应关系,基于接近度;

·估计将对的第一个成员最佳映射到第二个成员的刚性变换;

·将该变换应用于第一个结构中的所有特征。

按下“Coregister”按钮后,您需要指定在sMRI图像中与您的M/EEG基准点对应的点。
如果您有更多的基准点(例如在EEG中,原则上任何电极都可以用作基准点),在第一步中,您需要选择要使用的基准点。
可以选择超过3个基准点,但不能少于3个。
然后,对于每个选定的M/EEG基准点,您需要通过以下三种方式之一来指定在sMRI图像中的对应位置:

·选择(select)——一些常用点的位置(如鼻根点和耳前点)以及CTF推荐的MEG基准点在SPM中是硬编码的。
如果您的基准点对应于这些点之一,您可以选择此选项,然后从列表中选择正确的点。

·输入(type)——在这里,您可以为基准点输入MRI坐标(1 × 3向量,以毫米为单位)。
如果您的基准点不在SPM的硬编码列表中,建议仔细在模板图像或归一化到模板的受试者自身图像上找到正确的点。
您可以通过打开图像并使用SPM的显示/图像功能来进行此操作。
然后,您可以记录MNI坐标,并在所有需要使用“type”选项进行的共注册中使用这些坐标。

·点击(click)——这里会显示一个结构图像,您可以点击正确的点。
此选项适用于“快速且简单”的共注册或尝试不同的选项。

在指定基准点时,您还有跳过当前基准点的选项,但请记住,您只能在最终指定超过3个基准点时才能跳过。
否则,共注册将失败。

在您指定基准点后,系统会询问是否使用头部形状点(如果有)。
对于EEG,建议始终选择“是”。
对于MEG,如果您使用基于受试者sMRI的头部模型,并且对3个基准点有精确的信息(例如通过使用维生素E胶囊标记的扫描),使用头部形状可能会带来不利影响。
在其他情况下,它可能会有所帮助,如同在EEG中一样。

共注册的结果将在SPM的图形窗口中显示。
在继续之前仔细检查结果非常重要。
在顶部图中,您将看到头皮、内颅骨和包含传感器及基准点的皮层网格。
对于EEG,确保传感器在头皮表面。
对于MEG,检查头部相对于传感器的位置是否合理,例如头部不应伸出传感器阵列之外。
在底部图中,传感器标签将以拓扑数组的形式显示。
检查顶部标签对应前方传感器,底部对应后方,左侧对应左侧,右侧对应右侧,并确保标签在拓扑上位于预期的位置。


\section{正向计算(forward)}

正向计算是指计算皮层网格上每个偶极子对传感器的影响。
结果是一个 N×M 矩阵,其中 N 是传感器的数量,M 是网格顶点的数量(在前一步中选择的)。
由于该矩阵可能非常大,因此它不会存储在头文件中,而是存储在一个单独的 *.mat 文件中,该文件名中包含 SPMgainmatrix,并写在与数据集相同的目录中。
这个矩阵中的每一列都是一个所谓的“导向场”,对应一个网格顶点。

导向场的计算使用了 Robert Oostenveld 开发的“forwinv”工具箱,SPM 与 FieldTrip 共享。
该计算基于麦克斯韦方程,并对头部的物理属性作出假设。
这些假设被称为“正向模型”。

“forwinv”工具箱支持不同类型的正向模型。
当您按下“Forward Model”按钮(在成功共注册后应启用)时,您可以根据数据集的模态选择几种头部模型。
我们目前推荐MEG使用单壳模型,EEG使用“EEG BEM”。
您也可以尝试其他选项并使用模型证据进行比较(见下文)。
第一次使用新结构图像的“EEG\ BEM”选项(以及第一次使用“Template”选项)时,将进行一次长时间的计算,以基于头部网格准备BEM模型。
BEM模型将保存为一个较大的 *.mat 文件,文件名以 \_EEG\_BEM.mat 结尾,存储在与结构图像相同的目录中(对于模板,存储在SPM的“canonical”子目录中)。
当头部模型准备好后,它将在图形窗口中与皮层网格和传感器位置一起显示,您应最后一次验证所有内容是否匹配良好。

实际的导向场矩阵将在下一步开始时计算并保存。
这是一个耗时的步骤,对于高分辨率网格来说时间更长。
如果不更改共注册和正向模型,则导向场文件将用于所有后续的反演。


\section{反演重建}

要开始反演重建,按下“Invert”按钮。
首先会让您在“Imaging”、“VB-ECD”和“DCM”之间进行选择。
对于基于经验贝叶斯方法来定位由EEG或MEG测量的诱发响应、诱发功率或诱发功率的重建,按下“Imaging”按钮。
其他选项在其他地方有更详细的解释。

如果您的数据集中有多个条件的试验,接下来您需要选择是将所有条件一起反演还是选择一个子集。
如果计划在条件之间进行统计比较,建议将条件一起反演。
如果只有一个条件,或在选择条件后,您将选择“Standard”和“Custom”反演之间。
选择“Standard”反演时,SPM将以默认设置开始计算。这对应于多稀疏先验(MSP)算法,应用于整个输入数据段。

如果想要微调反演参数,请选择“Custom”选项。
然后可以在不同的超先验模型(IID - 相当于经典最小范数,COH - 类似于LORETA的方法的平滑先验)或MSP方法之间选择。

接着可以选择反演时间窗。
如果感兴趣的活动幅度较低,建议将时间窗限制在感兴趣的活动上,因为如果包含了不相关的高幅度活动,源重建方案将会专注于减少重建该活动的误差,而可能忽略感兴趣的活动。
在其他情况下,如果感兴趣的峰值是最强峰值或其幅度与其他峰值相当,最好不要限制时间窗,以便让算法对生成响应的所有脑源进行建模,然后使用适当的对比度聚焦于感兴趣的源(见下文)。
还有一个选项是对通道时间序列应用汉宁窗以减少试验开始和结束时的可能基线噪声。
此外,还可以预过滤数据。
最后,可以通过加载包含感兴趣区域MNI坐标的K×3矩阵的*.mat文件来限制解决方案到特定脑区。
这一选项可能起初显得奇怪,因为它似乎会过度偏向返回的源重建。
然而,在贝叶斯反演框架中,可以使用贝叶斯模型比较来比较同一数据的不同反演。
通过限制解决方案到特定脑区,可以大大简化模型,如果这种简化确实捕捉到生成响应的源,那么受限模型将比不受限模型有更高的模型证据。
然而,如果您建议的源不能解释数据,这种限制将导致模型拟合更差,并且视其差的程度,可能不受限模型在比较中会更好。
因此,使用此选项并进行后续模型比较是一种方法,例如,将文献或fMRI/PET/DTI的先验知识整合到反演中。
它还允许比较替代先验模型。

请注意,为使模型比较有效,影响输入数据的所有设置,如时间窗、使用的条件和过滤应保持相同。

SPM成像源重建还支持多模态数据集。
这些是同时记录的EEG和MEG数据集。
来自“Neuromag”MEG系统的数据集也被视为多模态。
如果数据集是多模态的,将出现一个对话框,要求从列表中选择源重建的模态。
如果选择多于一种模态,将进行多模态融合。
基于Henson等人的论文,该选项使用启发式方法重新调整不同模态的数据,以便一起使用。

一旦反演完成,您将在图形窗口的顶部图中看到最大活动区域的时间过程。
底部图将显示最大激活时间的最大强度投影(MIP)。
您还将看到可以用于模型比较的对数证据值,如上所述。
请注意,反演的所有输出并不会全部显示。
完整的输出包括整个时间窗内所有源和条件的时间过程。
可以使用3D GUI右下角的控件查看更多结果。
这些控件允许聚焦于特定时间、脑区和条件。
还可以显示神经活动演变的电影。


\section{总结反演重建结果为图像}

SPM提供了将反演重建结果写为3D NIfTI图像的功能,以便您随后可以使用基于GLM的统计分析进行随机场理论分析。
这类似于fMRI的二级分析,用于推断关于区域和试验特定效应(在被试间水平)的问题。

这涉及使用单个3D图像在源空间中总结试验和被试特定的响应。
关键是为每个对比图像创建一个时频对比窗口。
这是指定您想要进行推断的数据特征的灵活且通用的方法(例如,300毫秒左右的伽马活动或80到120毫秒之间的平均响应)。
这种对比通过按下Window按钮来指定。
然后系统会询问您感兴趣的时间窗口(以毫秒为单位,围绕刺激时间)。
可以指定一个或多个时间段(用分号分隔)。
要指定单个时间点,请重复同一个值两次。
下一个问题是关于频带的。
如果您只想平均源时间过程,将其保持为默认值零。
在这种情况下,窗口将按高斯加权。
如果指定特定频率或频带,则会生成一系列Morlet小波投影器,总结感兴趣的时间窗口和频带中的能量。

指定感兴趣频带为零与指定覆盖数据全部频率范围的宽频带之间存在差异。
在前一种情况下,每个偶极子的时间过程将按高斯加权平均。
因此,如果在您的时间窗口内该时间过程极性变化,活动可能会相互抵消,理想情况下,即使强响应也会产生零值。
在后一种情况下,功率在整个频谱上积分,忽略相位,这相当于在时域中计算平方振幅的总和。

最后,如果数据文件是分段的而不是平均的,您可以在“evoked”(诱发)、“induced”(诱导)和“trials”(试验)之间选择。
如果您有多个试验用于某些条件,则在前一步生成的投影器可以应用于每个试验并平均结果(诱导)或应用于平均试验(诱发)。
因此,可以定位与刺激无相位锁定的诱导活动。
也可以使用“evoked”选项聚焦于ERP的频率内容。
显然,结果会有所不同。
操作完成后,将显示您指定的投影器(底部图)和生成的最大强度投影(MIP)(顶部图)。
“trials”选项使每次试验可以导出一个图像,这对于在被试内进行统计分析可能有用。
图像以4D-NIfTI格式导出,每个条件一个文件,包含该条件的所有试验。

Image按钮用于写出对比结果。
可以将其导出为网格上的值(GIfTI)或体积3D图像(NIfTI)。
SPM统计机制支持这两种格式。
当生成每次试验的图像时,图像以4D-NIfTI格式导出,每个条件一个文件,包含该条件的所有试验。
导出的图像值经过归一化以减少被试间方差。
因此,为获得最佳结果,建议在一次步骤中导出将在同一统计分析中包含的所有时间窗口和条件的图像。
请注意,由源重建导出的图像由于从2D皮层片到3D体积的平滑处理而有些特殊。
SPM统计机制已优化以处理这些特殊情况并获得合理结果。
如果尝试使用旧版SPM或其他软件包分析这些图像,可能会得到不同(较不集中的)结果。


\section{渲染界面}

按下Render按钮可以打开一个新的GUI窗口,该窗口将在大脑表面上显示反演结果的渲染图。
您可以旋转大脑,聚焦于不同的时间点,运行电影,并比较预测和观察到的头皮拓扑图和时间序列。
一个有用的选项是“虚拟电极”,它允许您从网格上的任何点提取时间过程,并在该点的最大激活时间显示最大强度投影(MIP)。
只需按下按钮并点击大脑中的任意位置即可。

在SPM M/EEG Review功能中还有一个额外的工具可用于查看结果。


\section{组反演}

在MSP反演中遇到的一个问题是,有时它的结果“太好”,以至于每个受试者的解非常集中,导致受试者之间激活区域的空间重叠不足,从而在受试者间对比中无法得到显著结果。
这可以通过平滑来改善,但平滑会降低空间分辨率,从而削弱使用能够产生集中解的反演方法的主要优势。

为了规避这个问题,我们提出了一种MSP方法的修改,实际上限制了所有受试者的激活源相同,只允许激活程度有所不同。
我们展示了这种修改使得在保持准确空间定位的同时,可以获得接近非集中的方法(如最小范数)的显著性水平。

组反演比单个反演产生更好的结果,因为它为病态的反问题引入了一个额外的约束,即所有受试者的反应应由同一组源解释。
因此,在对整个研究进行分析并随后进行图像的GLM分析时,这应该是您的首选方法。

组反演的工作方式与上述描述非常相似。
您可以在打开3D GUI后按下“Group inversion”按钮启动它。
系统会要求您指定一组要一起反演的M/EEG数据集。
然后,例程会要求您为每个文件执行配准并预先指定所有反演参数。
还可以预先指定对比参数。
然后,反演将继续计算所有文件的反演解,并写出输出图像。
每个受试者的结果也将保存在相应输入文件的头文件中。
反演后,可以将该文件加载到3D GUI中并按照上述描述探索结果。


\section{批量源重建}

您可以使用SPM批处理工具运行成像源重建。
可以通过在主SPM窗口中按下“Batch”按钮,然后在“SPM”下的“M/EEG”中进入“M/EEG source reconstruction”来访问。
那里有单独的工具用于构建头模型、计算反演解以及计算对比和生成图像。
这使得例如可以从同一反演中生成多个不同对比的图像成为可能。
所有三个工具都支持多个数据集作为输入。
在反演工具的情况下,多个数据集将进行组反演。


\section{附录:数据结构}

在SPM中描述给定EEG/MEG数据集的Matlab对象被表示为D。
在该结构中,每个新的反演分析将由子结构字段D.inv中的一个新单元描述,并由以下字段组成:

·method: 字符串,指示方法,目前的情况是“ECD”或“Imaging”;

·mesh: 子结构,包含与源空间和头部建模相关的变量和文件名;

·datareg: 子结构,包含将EEG/MEG数据注册到MRI空间的相关变量和文件名;

·forward: 子结构,包含前向计算的相关变量和文件名;

·inverse: 子结构,包含相关变量、文件名以及结果文件;

·comment: 用户提供的字符串,用于描述当前分析;

·date: 上次对该分析进行修改的日期;

·gainmat: 增益矩阵文件的名称。