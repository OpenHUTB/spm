\chapter{传感器空间分析}
\label{ch:13}


本章描述了如何对 EEG/MEG 数据进行统计分析。
这需要将数据从 SPM M/EEG 格式转换为图像文件(NIfTI 格式)。
一旦数据转换为图像格式,M/EEG 的分析过程在程序上与 fMRI 的二级分析相同。
因此,关于最后一步的详细信息,请参阅 fMRI 部分。

在下拉菜单中的“Images”菜单中,选择“Convert to images”功能。
这将打开用于转换为图像的批处理工具。
您需要选择输入数据集,它可以是磁盘上的 mat 文件,也可以是先前处理步骤的依赖项。

然后,您需要设置转换的“模式”。
一般情况下,M/EEG 数据可以是五维的(3 个空间维度、时间和频率)。
SPM 统计工具只能处理最多三维数据。
虽然这是一个纯粹的实现限制,并且 SPM 方法背后的理论可以扩展到任何维度,但实际上高维统计结果可能非常难以解释,主要是因为我们人类无法可视化它们。
此外,不受约束的高维测试会因多重比较而受到非常严重的惩罚,因此在大多数情况下应避免使用。
因此,我们的目的是将数据维度减少到三维或以下。
传感器所在的三个空间维度可以通过将其位置投影到一个平面上减少为两个。
进一步减少维度将涉及对其中一个维度进行平均。
“模式”选项的选择对应于在子集数据维度上进行平均的所有不同可能性。
其中一些选项仅与存在频率维度的时频数据相关。

“Conditions”选项使得可以仅转换文件中部分条件的数据。
这对于构建批处理管道尤其有用。
转换模块会作为依赖项输出所有生成的 NIfTI 图像的列表。
这些图像可以用作后续步骤(例如统计设计规范)的输入。
通过在批处理中多次包含“Convert2images”模块,每个条件都可以有一个单独的依赖项,并在统计设计中的不同位置进入(例如,用于两个试验组之间的双样本 t 检验)。

“Channels”选项使得可以选择转换时的部分通道。
这些通道可以按模态(例如“EEG”)选择,或者按名称选择,或通过 mat 文件中的列表选择(例如,平均所有枕叶通道的数据)。

“Time window”和“Frequency window”选项限制了转换数据的范围,这在数据在此范围内被平均时尤其重要。
确保您只包括感兴趣的范围。

最后,“Directory prefix”选项指定了写出图像的目录前缀。
从同一数据集中生成多个不同组的图像(例如,不同模态或不同通道组),这一点非常重要。


\subsection{输出}

运行工具时,会在数据集位置创建一个目录。
其名称将是数据集名称加上指定的前缀。
在这个目录中,每个条件都会有一个对应的 nii 文件。
对于平均数据集,这些文件将是3D图像(某些维度的大小可能为1)。
对于分段数据集,将有4D-NIfTI图像,其中每一帧包含一个试验。

平均时间或频率:
虽然可以在转换为图像时直接创建在时间或频率维度上平均的2D头皮图像,但也可以通过对先前创建的3D图像的Z维度的一部分进行平均来生成。
这是通过“Images”菜单中的“Collapse time”工具来实现的。

掩码:
在设置统计分析时,使用显式掩码限制分析到感兴趣的固定时间窗口可能是有用的。
这种掩码可以通过从“Images”下拉菜单中选择“Mask images”来创建。
您将被要求提供一个未平滑的图像作为掩码模板,这可以是您导出的任何图像。
接下来,您需要指定感兴趣的时间(或频率)窗口以及输出掩码文件的名称。
此文件可以在统计设计中的“Explicit mask”选项中使用,或者在“Results”GUI中按下“small volume”按钮并选择“image”选项来指定体积时使用。


\subsection{平滑}

在使用下拉菜单中的“Images”中的“Smooth images”功能进行二级分析之前,必须对从 M/EEG 数据生成的图像进行平滑处理。
平滑处理是必要的,以适应受试者之间的空间/时间变异性,并使图像更符合随机场理论的假设。
平滑核的维度以原始数据的单位指定(例如,空间-时间为[mm mm ms],时频为[Hz ms])。
决定平滑程度的一般指导原则是匹配滤波器理论,该理论认为平滑核应匹配要增强的数据特征。
因此,平滑核的空间范围应大致与您要寻找的偶极子模式的范围相似(可能是几个厘米的量级)。
实际上,您可以尝试使用根据上述原则设计的不同核来平滑图像,并观察哪种效果最好。
在时间维度上的平滑并不总是必要的,因为数据滤波具有相同的效果。
对于头皮图像,您应该将“Implicit masking”选项设置为“yes”,以便在分析中继续排除头皮以外的区域。

一旦图像被平滑处理,您可以继续进行二级分析。