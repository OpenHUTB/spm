

\chapter{时间对齐}

纠正切片之间图像采集时间的差异。


切片时间校正的文件前面有一个“a”。


注意:指定切片获取顺序的 sliceorder arg 是 N 个数字的向量,其中 N 是每个体积的切片数。
每个数字指的是切片在图像文件中的位置。
矢量中的数字顺序是获取这些切片的时间顺序。 
要检查图像文件中切片的顺序,请使用 SPM 显示选项并将十字准线移动到 z=1 的体素坐标。
这对应于体积的第一个切片中的一个点。


该功能纠正切片采集时间的差异。
此例程旨在纠正回波平面扫描期间使用的切片采集的交错顺序。
必须进行校正以使每个切片上的数据对应于同一时间点。
在没有校正的情况下,一个切片上的数据将代表一个时间点,该时间点距离相邻切片(在交错序列的情况下)远为 TR 的 1/2。


此例程会及时“移动”信号以提供输出向量,该向量表示稍后或较早开始采样的相同(连续)信号。
这是通过构成信号的正弦波相位的简单移动来实现的。
回想一下,傅里叶变换允许将任何信号表示为不同频率和相位的正弦波的线性组合。
实际上,我们将为每个频率的相位添加一个常数,及时移动数据。




移位器——这是一个滤波器,信号将通过该滤波器进行卷积以引入相移。
它是在傅立叶域中明确构造的。
在时域中,它可以被描述为一个脉冲(delta 函数),它在时间上偏移了 TimeShift 描述的量。
校正通过使用 sincinterpolation 滞后(向前移动)每个切片上的时间序列数据来进行。
这导致每个时间序列都具有在切片与参考切片同时获取的情况下本应获得的值。
为了明确这一点,考虑在两个相邻切片上同时发生的神经事件(以及随之而来的血流动力学反应)。
从时间零开始获取切片“A”的值,与神经事件同时进行,而切片“B”的值在一秒后获取。
如果不进行校正,“B”值将描述一种血液动力学反应,该反应似乎在“B”切片上比在切片“A”上早一秒开始。
为了纠正这个问题,“B”值需要向右移动,即向最后一个值移动。



此校正假定数据是带限的(即,在高于奈奎斯特频率的频率下,数据中不存在有意义的信息)。
Josephs 等人(1997,Human Brain Mapping)[71] 的研究支持这一假设,该研究以 166 毫秒的有效 TR 获得了事件相关数据。
在高于我们典型的奈奎斯特 (0.25 HZ) 的频率下,没有出现生理信号变化。


使用切片时序校正时,输入正确的切片顺序非常重要,如果有任何不确定性,我们鼓励用户与他们的物理学家合作来确定实际的切片采集顺序。


还可以考虑通过在通知基组中包含时间导数而不是切片计时来增强模型,这可以解释 +/- 1 秒的计时变化。


由西北大学的 Darren Gitelman 于 1998 年撰写。(大部分)基于 ACQCORRECT。
来自宾夕法尼亚大学的 Geoff Aguirre 和 Eric Zarahn 的 PRO。


\section{数据}
主题或会话。
下面指定的相同参数将应用于所有会话。


\subsection{会话}
选择要切片时间校正的图像。


\section{切片数}
输入切片数。


\section{TR}
输入 TR(以秒为单位)。


\section{TA}

输入 TA(以秒为单位)。
它通常计算为 TR-(TR/nslices)。


您可以简单地输入此方程式,并将变量替换为适当的数字。
如果接下来的两项以毫秒为单位输入,则此条目将不会被使用,可以设置为 0。


\section{切片顺序}
输入切片顺序。

Bottom slice = 1. 序列类型和输入代码示例如下: 

ascending (first slice=bottom): [1:1:nslices] descending (first slice=top): [nslices:-1:1] interleaved ( middle-top): 

对于 k = 1:nslices round((nslices-k)/2 + (rem((nslices-k),2) * (nslices - 1)/2)) + 1, end interleaved (bottom - > up): [1:2:nslices 2:2:nslices] interleaved (top -> down): [nslices:-2:1, nslices-1:-2:1] 或者,您可以以毫秒为单位输入切片时间 对于每个切片单独。


如果这样做,下一项(参考切片)将包含参考时间(以毫秒为单位)而不是参考切片的切片索引。 对于西门子扫描仪,这可以在 MATLAB 中从 dicom 标头获取,如下所示(使用第一个之后的任何卷): $ hdr = spm_dicom_headers('dicom.ima'); slice_times = hdr1.Private_0019_1029 $ 请注意,假定切片顺序是从脚到头。 
如果不是,请输入:$ TR - INTRASCAN_TIME - SLICE_TIMING_VECTOR $


\section{参考切片}

输入参考切片。

如果在前一项中提供切片时间而不是切片索引,则此值应表示参考时间(以毫秒为单位)而不是参考切片的切片索引。

\section{文件名前缀}

指定要添加到切片时间校正图像文件的文件名之前的字符串。 默认前缀是“a”。
